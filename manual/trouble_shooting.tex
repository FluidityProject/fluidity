\chapter{Troubleshooting}

\section{Backtraces}

When reporting problems with the model it is often useful to obtain a backtrace
identifying the source code location of an error. If the model terminated via
an internal abort command (an FLAbort) then a backtrace is output to standard
error:

\begin{lstlisting}[language = Bash]
*** FLUIDITY ERROR ***
Source location: (State.F90, 1528)
Error message: Density is not a field name in this state
Backtrace will follow if it is available:
../../bin/dfluidity-debug(fprint_backtrace_+0x1f) [0x81a7bb5]
../../bin/dfluidity-debug(__fldebug_MOD_flabort_pinpoint+0x2e0) [0x816d575]
../../bin/dfluidity-debug(__state_module_MOD_extract_from_one_scalar_field+0xb23) [0x822448f]
../../bin/dfluidity-debug(__momentum_equation_MOD_solve_momentum+0x3983) [0x8779c1c]
../../bin/dfluidity-debug(__momentum_equation_MOD_momentum_loop+0x970) [0x8785bdc]
../../bin/dfluidity-debug(__fluids_module_MOD_fluids+0xf1f4) [0x819b7fc]
../../bin/dfluidity-debug(mainfl_+0x155) [0x816cf7d]
../../bin/dfluidity-debug(main+0x301) [0x8161931]
/lib/tls/i686/cmov/libc.so.6(__libc_start_main+0xe5) [0xb5c57685]
../../bin/dfluidity-debug [0x8161541]
Use addr2line -e <binary> <address> to decipher.
Error is terminal.
\end{lstlisting}

Line numbers for each call in the backtrace can be obtained via:

\begin{lstlisting}[language = Bash]
addr2line -e dfluidity-debug address
\end{lstlisting}

where \lstinline[language = Bash]*address* is one of the addresses in square
brackets at the end of a line of the backtrace.

If the model aborts via some other error, such as a Fortran run-time error or
a segmentation fault, then a backtrace can be obtained with gdb. First make
sure you have compiled with debugging, enabled at configure time with the
\lstinline[language = Bash]*--enable-debugging* argument. Then run
\fluidity\ in the debugger:

\begin{lstlisting}[language = Bash]
gdb dfluidity-debug
\end{lstlisting}

Run the debugger command ``break exit'':

\begin{lstlisting}[language = Bash]
(gdb) break exit
Function "exit" not defined.
Make breakpoint pending on future shared library load? (y or [n]) y
Breakpoint 1 (exit) pending.
\end{lstlisting}

to abort on any exit calls, and run the model:

\begin{lstlisting}[language = Bash]
(gdb) r arguments
\end{lstlisting}

where \lstinline[language = Bash]*arguments* are the arguments passed to
\fluidity. If the debugger breaks due to an exit command or segmentation fault
you can retrieve a backtrace with:

\begin{lstlisting}[language = Bash]
(gdb) bt
\end{lstlisting}

To obtain a backtrace for a parallel simulation you must launch gdb in parallel,
e.g.:

\begin{lstlisting}[language = Bash]
mpiexec -n 4 xterm -e gdb dfluidity-debug
\end{lstlisting}

This will launch multiple instances of the debugger. Run the commands above in
each instance to retrieve a backtrace for each process.
