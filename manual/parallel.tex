\chapter{Parallel}

\fluidity\ is a fully-parallel program, capable of running on thousands of processors. 
It uses the Message Passing Interface (MPI) library to communicate information between 
processors. Running \fluidity\ in parallel requires that MPI is available on your system
and is properly configured. This chapter assumes that this is the case. The 
terminology for parallel processing can sometime be confusing. Here, we use the 
term \emph{processor} to mean one physical CPU core and \emph{process} is one part
of a parallel instance of \fluidity. Normally, the number of processors used matches
the number of processes, i.e. if \fluidity\ is split into 256 parts, it is run on 256 processors.

\section{Set up}

The first step in running a \fluidity\ set-up in parallel is to create the software
use to decompose the initial mesh into multiple parts, fldecomp. This can be made using:
\begin{lstlisting}[language=bash]
make fltools
\end{lstlisting}
inside your \fluidity\ folder.The fldecomp binary will then be created in the bin/ directory. 
You can then decompose the initial mesh into multiple regions, one per process using the following command
\begin{lstlisting}[language=bash]
fldecomp -i triangle -n [PARTS] [BASENAME]
\end{lstlisting}
where BASENAME is the triangle mesh base name (excluding extensions). "-i triangle"
instruct fldecomp to perform a triangle-to-triangle decomposition. 
This will create PARTS partition triangle meshes together with PARTS .halo files. 

\section{Running}

To run in parallel there are no further changes needed, apart from running \fluidity
within the mpirun or mpiexec framework. Simply prepend the normal command line with mpiexec:
\begin{lstlisting}[language=bash]
mpiexec -n [PROCESSES] dfluidity -l -v2 my_setup.flml
\end{lstlisting}

