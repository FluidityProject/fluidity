\documentclass[11pt,a4paper]{article}
\usepackage{dsfont}
\usepackage{graphicx}
\usepackage{subfigure}
\usepackage{pdflscape}
\usepackage{float}
\usepackage{natbib}
\usepackage{amsmath}
\usepackage[top=2cm,bottom=2cm,left=2.2cm,right=2.2cm]{geometry}
\usepackage{setspace}
\usepackage{indentfirst}
\usepackage[font=small,format=plain,labelfont=bf,up,%% textfont=it
  ,up]{caption}
\usepackage{tikz}
\usetikzlibrary{calc,3d}
\setlength{\parindent}{10pt}
\setlength{\parskip}{7pt}
\usepackage{upgreek}
\usepackage{mathrsfs}
\usepackage{breqn}

\begin{document}

\begin{dmath}
  \int_{\Omega_h} {\bf g_h} \cdot {\bf z_h} \; \text{d}\Omega - \int_{\Omega_h} {\bf g_h} \cdot \nabla u_h \; \text{d}\Omega + \int_{\Sigma^0_h} \left\{ {\bf g_h} \right\} \cdot \mathscr{J}^0 u_h \; \text{d}\sigma = 0
\end{dmath}

\begin{dmath}
  \sum_B \left[ {\bf N_A} \cdot {\bf N_B} \right] {\bf z_h} - {\bf N_A} \cdot \nabla u_h + \left( \left\{ {\bf N_A} \right\} \cdot \mathscr{J}^0 u_h \right)_{\Sigma^0_h} = 0
\end{dmath}

\begin{dmath}
  \sum_B \left[ {\bf N_A} \cdot {\bf N_B} \right] {\bf z_h} = {\bf N_A} \cdot \nabla u_h - \left( \left\{ {\bf N_A} \right\} \cdot \mathscr{J}^0 u_h \right)_{\Sigma^0_h}
\end{dmath}

% \renewcommand\arraystretch{2}
% \begin{dmath}
%   {\bf C^e_{ab}} = \int_{\Omega_e}
%   \begin{bmatrix}
%     N_1 u \frac{\partial N_1}{\partial x} & N_1 u \frac{\partial N_2}{\partial x}  \\
%     N_2 u \frac{\partial N_1}{\partial x}  & N_2 u \frac{\partial N_2}{\partial x}
%   \end{bmatrix} \; dx  = \frac{hu}{2} \int_{-1}^1
%   \begin{bmatrix}
%     N_1 \frac{2}{h} \frac{\partial N_1}{\partial \xi} & N_1 \frac{2}{h} \frac{\partial N_2}{\partial \xi} \\
%     N_2 \frac{2}{h} \frac{\partial N_1}{\partial \xi} & N_2 \frac{2}{h} \frac{\partial N_2}{\partial \xi} 
%   \end{bmatrix} \; d\xi  = u \int_{-1}^1
%   \begin{bmatrix}
%     N_1 \frac{\partial N_1}{\partial \xi}  & N_1 \frac{\partial N_2}{\partial \xi} \\
%     N_2 \frac{\partial N_1}{\partial \xi}  & N_2 \frac{\partial N_2}{\partial \xi} 
%   \end{bmatrix} \; d\xi  = \frac{u}{2} 
%   \begin{bmatrix}
%      -1 & +1 \\
%      -1 & +1
%   \end{bmatrix}
% \end{dmath}

\end{document}
