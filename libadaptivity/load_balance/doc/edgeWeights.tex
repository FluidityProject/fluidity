\documentclass[11pt]{article}

\pagestyle{empty}                         %%%% No page Numbering
 
% Mathews packages....
\usepackage{amsbsy,amsmath,amssymb,psfig} 
\usepackage{times,mathpi,mathptm} 
\usepackage{graphicx,psfrag,rotating,subfigure} 

\begin{document} 
%\DeclareGraphicsExtensions{.pdf}

\date{}   
	
\title{Biasing graph partitioning ({\it against partitioning regions of interest})}

%\author{
%Gerard Gorman             \\
%T. H. Huxley School        \\
%Imperial College            \\
%Prince Consort Road          \\
%London SW7 2BP, UK            \\
%}

\maketitle
\thispagestyle{empty}
        
%%%%%%%%%%%%%%%%%%%%%%%%%%%%%%%%%%%%%%%%%%%%%%%%%%%%%%%%%%%%

%cpain wrote:
%> I read through your document and it looks good.
%> The points I'd like to make are:
%> 1) we need to take into account the node weight for balancing
%> partition load when we have adapted the mesh. We can use the
%> same criteria that we ahve used befor - in our paper.
%g This would be nice but there is a problem. According to Adrian,
%g there is no clear relation between the number of expected nodes and
%g the time that will be required for adaption. 
%> 2) we need to discurroage partition boundaries from
%> occuring where they occured on the pevious iteration.
%> Otherwise we will never adapte along partition boundaries.
%g Done. See the section ``Constraints on Adaptivity in Parallel''
%> 3) using a large edge weight around areas of high
%> curvature will discourage partition boundaries from occureing
%> across areas which will require a lot of work for mesh adaptivity.
%g Same as previous.
%> 4) the curvature normal to the boundary between subdomains
%> might be important
%> for the iterative convergence of the the DDM solution method.
%g hmm..I'm still thinking about this one.

\section{Constraints on Adaptivity in Parallel}
The mesh optimization procedure used in this work is essentially
serial. That is to say that it operates strictly on a local
basis. When the procedure is used in a parallel context, halo elements
are marked as being static and thus cannot be changed by the adaptive
procedure. This is necessary in order to keep the mesh consistent
across the distributed domain. The problem is that some of these
static elements may need to be adapted.

This restriction is overcome by preforming a series of adaptations and
repartitions. It is necessary to ensure that successive graph
partitions do not continue to bisect elements of interest. In actual
fact there is a strong tendency for the graph partitioning to follow
the previous partitioning in instances where the halo elements need to
be refined. This is a natural consequence of the fact that a partition
through an unrefined region will have a low edge cut. Of course the
inverse is true in the case where the mesh is being coarsened and thus
no problem arises. If it can be ensured that successive partitions can
avoid a region then the maximum number of adapt/repartition steps that
would be required is the same as the mesh dimensionality of the
problem.

A precursor to biasing a graph-partition calculation is the
assignment of {\it edge-weights} to the graph, where the graph
partitioning seeks to minimize the weight of the edge-cut.  In this
work, the edge-weight is derived directly from the mesh-adaptivity
formulation~\cite{adapt}. First the element functionals are calculated
over the whole domain. In the case of a tetrahedron, the element
functional is
\begin{equation}
F_e = \frac{1}{2}\sum_{l=1}^{6}\left(1.0 - r_l \right)^2 + \left( 1.0 - \frac{\alpha}{\rho_e} \right)^2
\end{equation}
where $F_e$ is the element functional for some element $e$, $\alpha =
1/(2\sqrt{6})$, and $\rho_e$ is the radius of the inscribed sphere of
element $e$. The scalar $\alpha$ is chosen such that $\alpha/\rho_e =
1$ for an ideal element. $r_l$ is the length of a edge $l$, with
respect to the metric $M(\Omega)$ defined in Euclidean space $\Omega$,
on an edge $l$. In general, the length $r_l$ with respect to
$M(\omega)$ is given by
\begin{equation}
r_l = \int_a^b \left( s_l(t)^T M(a + s_l(t)) s_l(t) \right)^{1/2} dt
\end{equation}
where $a$ and $b$ are the positions of the two nodes that define the
edge $l$, $s_l(t)$ is the position in $\Omega$, relative to the point
a, along edge l. In this work we are only concerned with meshes with
straight edges. Thus $r_l$ can be expressed using a one point
quadrature
\begin{equation}
r_l = \left(v_l^T M_l v_l \right)^{1/2}
\end{equation}
where $v_l = \pmb{b} - \pmb{a}$, and $M_l$ is the average metric of
nodes $a$ and $b$.
Armed with an expression for the element-functional, the edge-weight is defined as
\begin{equation}
w_l = MAX_{\forall e|l\in e}F_e
\end{equation}
The weight, when assigned in this fashion, tried to dis-encourage an
edge-cut if any one of its' elements requires adjustment.

Given the above definition for the edge-weights, two further
adjustments must be made before it can be used with a
graph-partitioning library such as the publicly available
ParMETIS~\cite{parmetis}. First the weights must be non-zero. The
element-functionals as defined above are zero for a ideal
element~\cite{adapt}. In addition, the weights are required to be
integers. With this in mind we define the edge-weight to be
\begin{equation}
w_l' = int(1.0 + w_l)
\end{equation}

\section{Minimizing the partitioning of regions of physical interest}
The existence of a partition in a local computational domain is
usually treated as being Dirichlet boundary. Convergence in DDM's may
degrade when the graph-partition lies along the curvature of a field
variable $\psi$. This suggests that for the solver, a good choice for
the edge-weights in the graph-partitioning calculation has to form
\begin{equation}
w_l = \left(\pmb{\nabla}\psi \right).\pmb{v}
\end{equation}

\bibliographystyle{plain}
\bibliography{migrate}

\end{document}






















